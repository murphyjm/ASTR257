\documentclass[12pt]{article}
\usepackage{epsfig}
\textwidth 7.5in                     % page width in inches
\textheight 9.9in                    % page height in inches
\topmargin -90pt                     % to fit on A4
\oddsidemargin -36pt                 % Left hand margin (odd pages)
\evensidemargin -28pt
\baselineskip 0.168in
\setlength{\parindent}{20pt}
%\def\etal{{\it et al. }}
\def\etal{et al.\ }
\newcommand{\msun}{\hbox{M$_{\odot}$}}

\begin{document}

\noindent
\centerline{\bf \textit{Nickel Observations with UCSC ASTR 257: Modern Astronomical Techniques}} 
\centerline{\bf PI A.\ Skemer}

\vskip 15pt

\centerline{\bf  Scientific Justification: }

As part of UC Santa Cruz Astronomy's new graduate curriculum, I am developing a new observing class, which consists of a week-long field trip to Lick Observatory.  The class is required of all first-year graduate students, and ensures that (1) our students will develop a broad range of observational skills at a time when opportunities to develop these skills are becoming more rare, and (2) our students will develop experience with Lick Observatory facilities, which will incentives them to become immediate scientific users.  This class has some parallels to the graduate observing workshop that has been run for many years at Lick Observatory, but the formal course will be more time intensive and will fulfill specific department requirements.

We are planning two observations with the Nickel 1-meter.  We will observe Pluto over two nights to determine its proper motion and we will observe an open cluster to make an HR diagram.  Both observations are mainstays of the PHYS/ASTR 136 undergraduate course.  We request two consecutive first-half nights for this program.

\newpage

\end{document}